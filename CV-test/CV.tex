% a mashup of hipstercv, friggeri and twenty cv
% https://www.latextemplates.com/template/twenty-seconds-resumecv
% https://www.latextemplates.com/template/friggeri-resume-cv

\documentclass[lighthipster]{simplehipstercv}
% available options are: darkhipster, lighthipster, pastel, allblack, grey, verylight, withoutsidebar
% withoutsidebar
\usepackage[utf8]{inputenc}
\usepackage[default]{raleway}
\usepackage[margin=0\paperwidth, a4paper]{geometry}


%------------------------------------------------------------------ Variablen

\newlength{\rightcolwidth}
\newlength{\leftcolwidth}
\setlength{\leftcolwidth}{0.23\paperwidth}
\setlength{\rightcolwidth}{0.75\paperwidth}

%------------------------------------------------------------------
\title{New Simple CV}
\author{\LaTeX{} Thomas}
\date{Septembre 2022}

\pagestyle{empty}
\begin{document}


\thispagestyle{empty}
%-------------------------------------------------------------

\section*{Start}

\simpleheader{headercolour}{Thomas}{Richard}{Developpeur Fullstack Node}{white}


%------------------------------------------------

% this has to be here so the paracols starts..
\subsection*{}
\vspace{5em}

\setlength{\columnsep}{0.05\paperwidth}
\columnratio{0.23}[0.75]
\begin{paracol}{2}
\hbadness5000
%\backgroundcolor{c[1]}[rgb]{1,1,0.8} % cream yellow for column-1 %\backgroundcolor{g}[rgb]{0.8,1,1} % \backgroundcolor{l}[rgb]{0,0,0.7} % dark blue for left margin

\paracolbackgroundoptions
% 0.9,0.9,0.9 -- 0.8,0.8,0.8

\footnotesize
{\setasidefontcolour
\flushleft

%%\begin{adjustwidth}{0.02\paperwidth}{}
%Thomas Richard\\
%Nationalité: Français\\
%Né 11/12/1995\\
%Vehiculé
%\end{adjustwidth}
%\bigskip
%\vspace{4em}

\bg{cvgreen}{white}{A propos de moi}\\[0.5em]
{\footnotesize
\begin{adjustwidth}{0.02\paperwidth}{}
    Je suis un passionné de sciences et de technologies.\\
    Pour cette raison, je souhaite développer mes compétences au sein de différents projets en tant que développeur Fullstack.\\
    En dehors du travail je suis toujours partant pour différentes activités listées ci-après!
    \end{adjustwidth}
    }
\bigskip
\vspace{4em}

\bg{cvgreen}{white}{Passe-temps}\\[0.5em]
\begin{adjustwidth}{0.02\paperwidth}{}
\texttt{Sciences}~/~\texttt{Technologies} \\
\texttt{Ski}~/~\texttt{Trek}~/~\texttt{Escalade}\\
\texttt{Jeux plateau}~/~\texttt{Jeux videos}
\end{adjustwidth}
\bigskip
\vspace{4em}

\bg{cvgreen}{white}{Social}\\[0.5em]
\infobubble{\faGlobe}{cvgreen}{white}{\href{https://minecraft.rayvol.net}{\scriptsize rayvol.net}}\\
\infobubble{\faGithub}{cvgreen}{white}{\href{https://github.com/samohtsp/Training}{\scriptsize Training}}\\
\infobubble{\faLinkedin}{cvgreen}{white}{\href{https://linkedin.com/in/thomas-richard-dev}{\scriptsize LinkedIn}}\\
\infobubble{\faAt}{cvgreen}{white}{\scriptsize thomas.richard@rayvol.net}\\

\phantom{turn the page}

\phantom{turn the page}
}
%-----------------------------------------------------------
\switchcolumn


\begin{minipage}[t]{0.3\textwidth}
\section*{Stack}
\begin{tabular}{l}
      \bg{skilllabelcolour}{iconcolour}{VsCode} \\
      \bg{skilllabelcolour}{iconcolour}{Git} \\
      \bg{skilllabelcolour}{iconcolour}{Linux} \\
      \bg{skilllabelcolour}{iconcolour}{WebPack}\\
      %\bg{skilllabelcolour}{iconcolour}{Angular} \\
      \bg{skilllabelcolour}{iconcolour}{Node.js} \\
      \bg{skilllabelcolour}{iconcolour}{PostgreSQL} \\
     
\end{tabular}
\end{minipage}
\begin{minipage}[t]{0.35\textwidth}
\section*{Programming}
\begin{tabular}{r @{\hspace{0.5em}}l}
     \bg{skilllabelcolour}{iconcolour}{Javascript} & Avancé\\
     \bg{skilllabelcolour}{iconcolour}{html, SASS (css)} & Intermédiaire \\
     \bg{skilllabelcolour}{iconcolour}{C\#} & Intermédiaire\\
     \bg{skilllabelcolour}{iconcolour}{C/C++} & Intermédiaire\\
     \bg{skilllabelcolour}{iconcolour}{SQL} & Débutant \\
     
     \bg{skilllabelcolour}{iconcolour}{python} & Intermédiaire \\
\end{tabular}
\end{minipage}
\vspace{1em}


\begin{minipage}[t]{0.3\textwidth}
\section*{Soft skills}
\begin{tabular}{>{\footnotesize}p{0.9\textwidth}}
    \textbf{Adaptation}\\
    \textbf{Communication}\\
    \textbf{Attentif}\\
    \textbf{Investi}\\
\end{tabular}
\bigskip
\end{minipage}
\begin{minipage}[t]{0.35\textwidth}
\section*{Langages}
\begin{tabular}{l | ll}
\textbf{English} & C1 & {\phantom{x}\footnotesize niveau professionnel} \\
\textbf{Français} & C2 & {\phantom{x}\footnotesize langue maternelle} \\
\textbf{Español} & B1 & {\phantom{x}\footnotesize niveau lycée}
\end{tabular}
\bigskip
\end{minipage}

\begin{minipage}[t]{0.65\textwidth}
\section*{Formations et Diplomes}
\begin{tabular}{r p{0.9\textwidth}}
\vspace{1em}
    \cvdegree{2018}{Master 2 Physique}{Aix-Marseille Universite}{Physique fondamentale}{Préparation à l'agrégation et la recherche.}
    
    \cvdegree{2017}{Master 1 Physique}{Aix-Marseille Universite}{Physique fondamentale}{C/C++(11): Simulation de sytème planétaire et de système quantique, puis améliorations de ces projets via optimisation et multi-threading.} \\
    \cvdegree{2013-2016}{Licence de Physique-chimie}{Aix-Marseille Universite}{Physique fondamentale}{Python 2 ans: SciPy, numpy, matplotlib sur Linux (Debian) modélisation de fluide et mécanique - C/C++(11) 2 ans: gcc, gnuplot sur Linux(Debian), analyse de données et premières approches de la compilation.} 
\end{tabular}
\end{minipage}
\bigskip

\begin{minipage}[t]{0.65\textwidth}
\section*{Experiences Professionnelles}
\vspace{10pt}

\begin{tabular}{r| p{0.9\textwidth}}
\cvevent{2019--2021}{Fullstack développeur}{SII}{Aix-en-Provence\color{cvred}}{
\vspace{8pt}
Consultant à Airbus Helicopters - Flight Perfo\\
Consultant à la Marine Nationale (CEPA) - ZEPHYR\\
Consultant à Airbus Helicopters - PTU}
\end{tabular}
\vspace{8pt}

\begin{tabular}{r| p{0.9\textwidth}}
\cvevent{2018--2019}{Professeur}{Education nationale}{Chalon-sur-Saône\color{cvred}}{
Professeur de physique et chimie en collège\\
Professeur de physique et chimie en lycée}
\end{tabular}
\vspace{8pt}

\begin{tabular}{r| p{0.9\textwidth}}
\cvevent{\hspace{23.2pt}2017}{Stage développeur}{CPPM}{Marseille\color{cvred}}{
Développeur logiciel d'analyse physique}
\end{tabular}
\vspace{8pt}

\begin{tabular}{r| p{0.9\textwidth}}
\cvevent{\hspace{0.2pt}2015--2016}{Professeur}{AMU}{Marseille \color{cvred}}{
Professeur en Travaux Dirigés}
\end{tabular}
\end{minipage}

%----------------------------------------------------------------------------------------
%----------------------------------------------------------------------------------------
%	Second Page
%----------------------------------------------------------------------------------------
%----------------------------------------------------------------------------------------

\begin{minipage}[t]{0.65\textwidth}
\vspace{1em}
\section*{Experiences}

\begin{tabular}{r p{0.9\textwidth}}
\persoevent{}{Webapp Minecraft}{ Fullstack developpeur}{}{
\begin{adjustwidth}{0.01\rightcolwidth}{}
\vspace{3pt}
\textbf{Minecraft-clone:} application Node qui reproduit un minecraft simplifié dans le navigateur.\\
Ce projet m'a permis de créer un environnement de travail complexe.\\
Environnement nécessaire, pour la maintenabilite, la production comme le développement.\\
\end{adjustwidth}

\vspace{3pt}
\begin{tabular}{ p{0.4\textwidth} | p{0.4\textwidth}}
\textbf{Environnement de Dev:}              &\textbf{Librairies:}\\    
-FrameWork: Nodejs                          &-Threejs : 3D web Renderer \\
-Serveur: Express.js                        & -Cannon-es: Moteur Physique pour le web\\
-ES6 transpilation: Babel                   &\textbf{Base de données:}\\
-Linter : ESlint                            &-MySQL\\
-Bundler: Webpack 5                         & \textbf{Tests: } \\
-Versionning: github                        &-Jest \\
-Uglify code: terser-webpack-plugin         & \\
-Builds: Dev et Prod serveur                & \\
\end{tabular}
}
\end{tabular}
\vspace{3pt}
 \par\rule{\textwidth}{0.3pt} 
 \par
\vspace{3pt}



\begin{tabular}{r p{0.9\textwidth}}
\persoevent{}{Galaxy background}{ Fullstack developpeur}{}{
\begin{adjustwidth}{0.01\rightcolwidth}{}
\textbf{Galaxy:} application Node dans laquelle on peut naviguer dans une galaxie de particules.\\
Dans ce projet, j'ai cherché à maîtriser les bases de la librairie Threejs pour une application from scratch.
\end{adjustwidth}
\vspace{3pt}
\begin{tabular}{p{0.4\textwidth} | p{0.4\textwidth}}
\textbf{Environnement de Dev:}              & \textbf{Librairies:}\\    
-FrameWork: Nodejs                          & -Threejs : 3D web Renderer \\
-Bundler: Webpack 5                         & \\
-Versionning: github                        & \\
-Build: Dev serveur                         & \\
\end{tabular}
}
\end{tabular}
\vspace{3pt}
 \par\rule{\textwidth}{0.3pt} 
 \par
\vspace{3pt}


\begin{tabular}{r p{0.9\textwidth}}
\persoevent{}{NFT Marketplace}{ Fullstack developpeur}{}{
\begin{adjustwidth}{0.01\rightcolwidth}{}
\textbf{NFT-Marketplace:} application Node/nextjs, dans laquelle on peut vendre/acheter/burn/mint des NFTs.\\
Dans ce projet, j'ai cherché à appliquer ce que j'ai appris au travers du tutoriel "crypto Zombie" de la fondation Ethereum.\\
\end{adjustwidth}
\vspace{3pt}
\begin{tabular}{ p{0.4\textwidth} | p{0.4\textwidth}}
\textbf{Environnement de Dev:}              &\textbf{Librairies:}    \\     
-FrameWork: Nodejs                          &-Threejs : 3D web Renderer \\ 
-Bundler: Webpack 5                         &-HardHat\\ 
-Linter : ESlint                            & \\
-Versionning: github                        & \\
-Build: Dev serveur                         & \\
\end{tabular}
}
\end{tabular}
\vspace{3pt}
 \par\rule{\textwidth}{0.3pt} 
 \par
\vspace{3pt}


\begin{tabular}{r p{0.9\textwidth}}
\persoevent{}{Portfolio}{ Frontend}{}{
\begin{adjustwidth}{0.01\rightcolwidth}{}
\textbf{Portfolio:} Site web basique html/SASS\\
L'objectif de ce projet était de coder et conserver un ensemble de concepts liés au SASS
\end{adjustwidth}
\vspace{3pt}
\begin{tabular}{ p{0.4\textwidth} | p{0.4\textwidth}}
\textbf{Environnement de Dev:}      & \\
-langages: html/SASS JavaScript     &  \\    
-Compiler: SASS compiler            & \\
-Serveur: liveServer                & \\
-Versionning: github                & \\
\end{tabular}
}
\end{tabular}
\vspace{3pt}
 \par\rule{\textwidth}{0.3pt} 
 \par 
\vspace{3pt}


\begin{tabular}{r p{0.9\textwidth}}
\persoevent{}{Flight Perfo}{ Fullstack developpeur}{}{
\begin{adjustwidth}{0.01\rightcolwidth}{}
\textbf{Flight Perfo:} application qui permet le calcul de perfomances d'un hélicoptère dans l'objectif de remplacer les manuels de vols\\
\end{adjustwidth}
\vspace{3pt}
\begin{tabular}{ p{0.4\textwidth} | p{0.4\textwidth}}
\textbf{Environnement de Dev:}          & Certification Process: DO-330\\
-Framework: Qt                          & Ticketing: Mantis\\    
-Frontend: Javascript Vanilla, html/css & \textbf{Tests:} \\
-Backend: calculateur en C++            & coco(code coverage) \\
-Versionning: tortoise SVN              & test unitaire sous Qt framework\\
\end{tabular}
\vspace{3pt}


Les inputs du client étaient faites en C\#\\
\begin{tabular}{ p{0.4\textwidth} | p{0.4\textwidth}}
\textbf{Environnement de Test:}         & Certification Process: DO-330\\
-Framework: .Net 4.5                    & Ticketing: Mantis\\ 
-Frontend: windows Form                 & \textbf{Tests:} \\
-Backend: calculateur en C\#            & Tests: visual studio 2019 testing tool\\
-Versionning: tortoise SVN              & \\
\end{tabular}
}
\end{tabular}
\vspace{3pt}
 \par\rule{\textwidth}{0.3pt} 
 \par
\vspace{3pt}


\begin{tabular}{r p{0.9\textwidth}}
\persoevent{}{ZEPHYR}{ Fullstack developpeur}{}{
\begin{adjustwidth}{0.01\rightcolwidth}{}
\textbf{ZEPHYR:} Application pour pilote d'essai\\
Application qui aide les pilotes d'essai à déterminer comment et quand ils peuvent atterrir ou décoller, avec un hélicoptère, d'un navire\\
\end{adjustwidth}
\vspace{3pt}
\begin{tabular}{ p{0.4\textwidth} | p{0.4\textwidth}}
\textbf{Environnement de Dev:}          & Certification Process: DO-330\\
-Framework: Qt                          & Ticketing: Mantis\\    
-Frontend: Javascript Vanilla, html/css & \textbf{Base de donees:} \\
-Backend: calculateur en C++            & SQLite \\
-Versionning: tortoise SVN              & -Support pour l'édition de spécifications\\\\
\end{tabular}
\textbf{Lead du projet :}  communication avec nos clients, division des tâches pour notre équipe, estimations des charges de travail.
}
\end{tabular}
\vspace{3pt}
 \par\rule{\textwidth}{0.3pt} 
 \par
\vspace{3pt}


\begin{tabular}{r p{0.9\textwidth}}
\persoevent{}{PTU}{ Developpement de Tests}{}{
\begin{adjustwidth}{0.01\rightcolwidth}{}
\textbf{PTU:} unitary test plan\\
Développement de tests unitaires en C dans l'objectif de vérifier le code source embarqué des hélicoptères.\\
\end{adjustwidth}
\vspace{3pt}
\begin{tabular}{ p{0.4\textwidth} | p{0.4\textwidth}}
\textbf{Environnement de Dev:}          & -Edition/correction de spécifications\\ 
-Framework: RTT                         & -Certification Process: DO-178.b\\
-Relecture des spécifications via DOORS & -Ticketing: Mantis\\  
-Vérification du code source sous SCADE & \\
-Versionning: tortoise SVN              & \\
\end{tabular}
}
\end{tabular}
\vspace{3pt}
 \par\rule{\textwidth}{0.3pt} 
 \par
\vspace{3pt}
\end{minipage}


\vfill{} % Whitespace before final footer
\vspace{6.2em}
%----------------------------------------------------------------------------------------
%	FINAL FOOTER
%----------------------------------------------------------------------------------------
\setlength{\parindent}{0pt}
\begin{minipage}[t]{0.9\rightcolwidth}
\begin{center}\fontfamily{\sfdefault}\selectfont \color{black!70}
{\small Thomas Richard 
\icon{\faPhone}{cvgreen}{} +337 69 57 86 67\\
\icon{\faAt}{cvgreen}{} {thomas.richard@rayvol.net}
}
\end{center}
\end{minipage}

\end{paracol}
\end{document}
